% !TEX program = xelatex

\documentclass{resume}
%\usepackage{zh_CN-Adobefonts_external} % Simplified Chinese Support using external fonts (./fonts/zh_CN-Adobe/)
%\usepackage{zh_CN-Adobefonts_internal} % Simplified Chinese Support using system fonts

\begin{document}
\pagenumbering{gobble} % suppress displaying page number

\name{Yansong Huang}

\basicInfo{
  \email{873994556@bupt.edu.cn} \textperiodcentered\ 
  \phone{(+86) 180-175-85587} \textperiodcentered\ 
  \homepage[https://iso497.github.io/]{https://iso497.github.io/}}

\section{\faGraduationCap\ Education}
\datedsubsection{\textbf{Beijing University of Posts and Telecommunications (BUPT)}, Beijing, China}{2022 -- Present}
\textbf{Master student} in Information and Communication Engineering, expected March 2025 \newline
\textbf{GPA (First Year):} 91.65/100 \newline
\textbf{Major Courses:} Theory of Communication Network (90), Graph Theory and Applications (91)

\datedsubsection{\textbf{Beijing University of Posts and Telecommunications (BUPT)}, Beijing, China}{2018 -- 2022}
\textbf{B.S.} in Telecommunications Engineering with Management \newline
\textbf{GPA:} 86.25/100 \newline
\textbf{Major Courses:} Principles of Communications (93), Internet Protocols (92)

\datedsubsection{\textbf{Queen Mary University of London (QMUL)}, Beijing, China}{2018 -- 2022}
\textbf{Joint Programme} with BUPT \newline
\textbf{B.S.} with Honors, First Class in Telecommunications Engineering with Management \newline
\textbf{GPA:} 86.25/100 \newline
\textbf{Major Courses:} Advanced Transform Methods(93), Interactive Media Design and Production (96)

\section{\faBook\ Publications}
\textbf{Papers} \newline
\textbf{Y. Huang}, X. Li, M. Zhao, M. Peng, "Asynchronous Federated Learning via Over-the-Air Computation in LEO Satellite Networks," \textit{IEEE Transactions on Wireless Communications}, Under Review.
\begin{itemize}
  \item This paper proposed an asynchronous federated learning (FL) framework in low-earth orbit (LEO) satellite networks by exploiting multiple high-altitude platforms for model aggregation, where the advanced over-the-air computation (AirComp) transmission scheme is utilized for the sake of further reducing energy consumption.
  \item To find the optimal solution of global FL model aggregation scheme and beamforming vector, this paper proposed a quantity-quality jointed linkage search algorithm combining depth-first search (DFS) and breadth-first search (BFS) algorithm with subtree pruning.
\end{itemize}
\textbf{Y. Huang}, H. Wei, J. Yang, M. Wu, "Damaged Road Extraction Based on Simulated Post-Disaster Remote Sensing Images," \textit{2021 IEEE International Geoscience and Remote Sensing Symposium IGARSS}, Brussels, Belgium, 2021, pp. 4684-4687.
\begin{itemize}
  \item To enlarge the numbers of post-disaster remote sensing images for related deep learning tasks, this paper applied CoCosNet on translating pre-disaster images to simulated post-disaster images of the same area. The work was varified effective by the high accuracy of applying D-LinkNet trained with real post-disaster images on detecting damaged roads in simulated post-disaster images. 
\end{itemize}

\textbf{Patents} \newline
X. Li, \textbf{Y. Huang}, M. Zhao, "A User-Centric Federated Learning Method and Device Based on Visible Light Communication," Beijing: CN202310244213.8, June 23, 2023. \newline
X. Li, \textbf{Y. Huang}, M. Zhao, "A Method, Device, System, and Virtual Node for Constructing a Digital Twin Network," Pending.



\section{\faTasks\ Projects}

\datedsubsection{\textbf{Large-Scale Distributed Mobile Ad-hoc Network Emulation System}}{Mar. 2022 -- Present}
This project constructed a communication \textbf{emulation} system deployed in a \textbf{distributed} framework and managed by Kubernetes. The system utilized Docker to create containers as independent nodes and emulated the effect of physical layer and data link layer through its highly flexible model plugin. 
\begin{itemize}
  \item Emulated Link 16 data link by TDMA model and calculated the BER-SINR curve in matlab as the input to the TDMA model as the judge of accepting a signal or not. 
  \item Took the tracks of UAV swarm generated by path planning algorithms dynamically as input and emulated the communication of UAV swarm in the scenarios with obstacles.
  \item Deployed data compression and depression algorithm at the application layer to realize efficient data transmission in the emulation system.
\end{itemize}

\datedsubsection{\textbf{Deep Learning Based Human State Assessment System}}{Oct. 2021 -- Present}
This project construct a \textbf{non-contact human state assessment} system to monitor the realtime state of drivers and alert when abnormal state is predicted through videos captured by only one or two cameras. This system including target detection, keypoint detection, head pose estimation and time series prediction algorithms.
\begin{itemize}
  \item Participated in the survey, design, implementation and testing of developing the system.
\end{itemize}

\datedsubsection{\textbf{High-Resolution Road Disaster Monitoring and Assessment System}}{Aug. 2020 -- Jun. 2021}
% \textbf{Brief introduction:} This project was proposed by National Disaster Reduction Center of the Ministry of Emergency Management of People's Republic of China and completed by Pattern Recognition and Intelligent Vision Laboratory of Beijing University of Posts and Telecommunications. It aimed to construct an artificial intelligent assisted system to monitor geological disasters and assess road damage through remote sensing images taken by satellites. 
This project aimed to construct an \textbf{artificial intelligent} assisted system to monitor geological disasters and \textbf{assess road damage} through remote sensing images taken by satellites. 
\begin{itemize}
  \item Processed remote sensing images with ArcGIS API for Python.
  \item Surveyed on the deep learning algorithms for detecting damaged road in post-disaster remote sensing images.
  \item Implemented and validated the feasibility of damaged road detection algorithm D-LinkNet.
  \item Applied CoCosNet on generating simulated post-disaster remote sensing images.
  \item Published the paper "Damaged Road Extraction Based on Simulated Post-Disaster Remote Sensing Images" as the first author.
\end{itemize}

\datedsubsection{\textbf{Intelligent Cloud Gallery}}{Jun. 2020 -- May 2021}
This project created a \textbf{digital photo album} that demonstrated photos, artworks, and dynamic images to users based on current environmental conditions, date, and user preferences and \textbf{matched music to pictures}, providing users with a diverse artistic perspective. The voice assistant answered questions about artworks and enabled users to control the photo album in voice.
\begin{itemize}
  \item Participated in requirements analysis, system design, implementation and testing of developing the product.
  \item Developed the digital photo album on Raspberry Pi.
  \item Developed the Android app that allowed users to control the digital photo album via their smartphones.
\end{itemize}

\datedsubsection{\textbf{The Guidance of Music Influence on Music Evolution}}{Feb. 2021}
This project analyzed the \textbf{music influence}, revealed the characteristics of the music change and the musical evolution and revolution from \textbf{big data}. 
\begin{itemize}
  \item Developed a mathematical model to create a directed network of musical influence and analyzed related characteristics.
  \item Implemented the mathematical model by programming and wrote the paper.
  \item Was designated as Meritorious Winner in 2021 Interdisciplinary Contest In Modeling.
\end{itemize}

% \datedsubsection{\textbf{Course Projects}}{2018-2022}
% These course projects were completed in my undergraduate courses, covering various programming languages. I worked as a designer and programmer in all of them and accomplished most work.
% \begin{itemize}
%   \item Fitness Builder Program (Java \& JavaFX)
%   \item Chinese Cuisine Website (JavaScript \& CSS \& HTML)
%   \item FTP Server (C)
%   \item Electronic Organ (C \& Arduino)
%   \item Telephone Management System (C)
% \end{itemize}

\section{\faCogs\ Skills}
\begin{itemize}[parsep=0.5ex]
  \item Programming Languages: Python > Java > Matlab > C > C++ > HTML + CSS + JavaScript
  \item Platform: Linux
  \item Tools: LXC, Docker, Kubernetes
\end{itemize}

\section{\faHeartO\ Honors and Awards}
\datedline{\textit{First-Class Scholarship}, Twice}{Oct. 2022, Oct. 2023}
\datedline{\textit{Outstanding Graduate}, Award on Undergraduate Graduation Ceremony}{Jun. 2022}
\datedline{\textit{QM Prize}, Award on Undergraduate Graduation Ceremony}{Jun. 2022}
\datedline{\textit{\nth{3} Prize}, Award on College Students Innovation and Entrepreneurship Forum}{Jun. 2021}
\datedline{\textit{Meritorious Winner}, Award on 2021 Interdisciplinary Contest In Modeling}{Apr. 2021}
\datedline{\textit{Third-Class Scholarship}, Twice}{Oct. 2019, Oct. 2021}

\end{document}
